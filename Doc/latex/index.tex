\href{https://travis-ci.com/ScottKolo/Mongoose}{\tt } \href{https://codecov.io/gh/ScottKolo/Mongoose}{\tt }

Mongoose is a graph partitioning library. Currently, Mongoose only supports edge partitioning, but in the future a vertex separator will be added.

\subsection*{Installation}

Mongoose uses C\+Make. To build Mongoose, follow the commands below\+:


\begin{DoxyCode}
1 git clone https://github.com/ScottKolo/Mongoose
2 cd Mongoose
3 mkdir \_build # Create a build directory
4 cd \_build 
5 cmake ..     # Use CMake to create the Makefiles
6 make         # Build Mongoose
\end{DoxyCode}


\subsection*{Usage}

T\+O\+DO\+: Write usage instructions

\subsection*{Contributing}


\begin{DoxyEnumerate}
\item Fork it!
\item Create your feature branch\+: {\ttfamily git checkout -\/b my-\/new-\/feature}
\item Commit your changes\+: {\ttfamily git commit -\/am \textquotesingle{}Add some feature\textquotesingle{}}
\item Push to the branch\+: {\ttfamily git push origin my-\/new-\/feature}
\item Submit a pull request \+:D
\end{DoxyEnumerate}

\subsection*{History}

T\+O\+DO\+: Write history

\subsection*{Credits}

The following people have made significant contributions to Mongoose\+:


\begin{DoxyItemize}
\item Nuri Yeralan, Microsoft Research
\item Scott Kolodziej, Texas A\&M University
\item Tim Davis, Texas A\&M University
\item William Hager, University of Florida
\end{DoxyItemize}

\subsection*{License}

T\+O\+DO\+: Write license 